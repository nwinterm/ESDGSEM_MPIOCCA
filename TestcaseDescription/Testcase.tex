\documentclass[times,sort]{elsarticle}


\usepackage{yfonts}

\usepackage[margin=2.0cm]{geometry}
%
\usepackage{graphicx}
\usepackage{amssymb}
\usepackage{amsmath}
% for algorithms
\usepackage[ruled,vlined]{algorithm2e}
\DontPrintSemicolon
\SetAlFnt{\small}

\usepackage{mathrsfs}
\usepackage{epstopdf}
\usepackage{stmaryrd}
\usepackage{subfig}
\usepackage{amsthm}
\usepackage{lmodern}
\usepackage{enumitem}
\usepackage{amsfonts}
\usepackage{mathtools}
\usepackage{booktabs}
\usepackage{xcolor}
\usepackage{stackrel}
\usepackage{accents}



\DeclareGraphicsRule{.tif}{png}{.png}{`convert #1 `dirname #1`/`basename #1 .tif`.png}
\theoremstyle{plain}
\newtheorem{thm}{Theorem}
\newtheoremstyle{proofpartstyle}
  {1pc} % Space above
  {1pc} % Space below
  {} % Body font
  {2.5mm} % Indent amount
  {\itshape} % Theorem head font
  {} % Punctuation after theorem head
  {-0.1em} % Space after theorem head
  {} % Theorem head spec (can be left empty, meaning `normal')
\theoremstyle{proofpartstyle} \newtheorem*{proofpart}{Proof of Part}
\makeatletter
\@addtoreset{proofpart}{theorem}
\makeatother
\theoremstyle{plain}
\newtheorem{Lem}{Lemma}
\newtheorem{Prop}{Proposition}
\theoremstyle{remark}
\newtheorem{rem}{Remark}

\allowdisplaybreaks
\renewcommand{\qedsymbol}{$\blacksquare$}

\usepackage{placeins}

\numberwithin{equation}{section}

%newcommands
%
\newcommand{\pderivative}[2]{\frac{\partial #1}{\partial #2}}
\newcommand{\mytext}[1]{\textnormal{\textbf{#1}}}

\newcommand{\dmat}{\mat{D}}
\newcommand{\dmatT}{\mat{D}^T}
\newcommand{\mmat}{\mat{{M}}}
\newcommand{\qmat}{\mat{{Q}}}
\newcommand{\diffmat}{\mmat^{-1}\boldsymbol{\Delta}}
\newcommand{\diffmatT}{\boldsymbol{\Delta}^T\mmat^{-1}}
\newcommand{\smat}{\mat{{S}}}
\newcommand{\zmat}{\mat{{Z}}}
\newcommand{\average}[1]{\overline{ #1}}
%\newcommand{\average}[1]{\left\{\!\left\{ #1\right\}\!\right\}}
\newcommand{\jump}[1]{\llbracket #1 \rrbracket}
\newcommand{\ip}[1]{\left\langle#1\right\rangle_N}

\newcommand{\half}{\frac{1}{2}}
\newcommand{\fourth}{\frac{1}{4}}
\newcommand{\mat}[1]{\mathbf{#1}}
% all just use one vector notation now to avoid confusion

\newcommand\statevec[1]{\vec{ #1}}% space vector, x\hat x + y \hat y
\newcommand\spacevec[1]{\vec{ #1}}% space vector, x\hat x + y \hat y
\newcommand\tensor[1]{\overset{\text{$\leftrightarrow$}}{#1}}% state vector, e.g. [rho rhou rhoe]^{T}
\newcommand{\arry}[1]{\mathbf{ #1}}

\newcommand{\hmat}{\mat{{h}}}
\newcommand{\umat}{\mat{{u}}}
\newcommand{\vmat}{\mat{{v}}}
\newcommand{\bmat}{\mat{{b}}}
\newcommand{\Bmat}{\mat{{B}}}
\newcommand{\Dhat}{\hat{{D}}}
\newcommand{\dtdx}{\frac{\Delta t}{\Delta x}}
\newcommand\acclrvec[1]{\accentset{\,\leftrightarrow}{#1}}  % define leftrightarrow as an accent
\definecolor{redcol}{rgb}{1.,0.,0.0}


%extra space between paragraphs
\parskip2mm

\begin{document}


%%%%%%%%%%%%%%%%%%%%%%%%%%%%%%%%%%%%%%%%%%%%%%%%%%%%%%%%%%%%%%%%%%%%
\section{Testcase included in the code}
%%%%%%%%%%%%%%%%%%%%%%%%%%%%%%%%%%%%%%%%%%%%%%%%%%%%%%%%%%%%%%%%%%%%
\subsection{TESTCASE 1: Periodic Convergence Test}
\subsection{TESTCASE 2: Well Balanced Test with Discontinous Bottom}
\subsection{TESTCASE 3: Entropy Glitch Test}
\subsection{TESTCASE 20: Partial Dam Break (CARTESIAN)}
Note: interior boundary conditions for partial dam are hacked in Mesh.cpp
\subsection{TESTCASE 30: Steep Dam Break - Test for Shock Capturing // Limiting}
\subsection{TESTCASE 31: 2D Oscillating Lake // Parabolic Bowl}
\subsection{TESTCASE 32: Flooding over 3 Mounds}
\subsection{TESTCASE 33: Dam Break over 3 Mounds}
\subsection{TESTCASE 34: 2D solitary wave runup}
\begin{equation}
\begin{aligned}
&h(x,y,0)=\max \left( 0, h_0 + \frac{A}{h_0}  \text{sech}^2 (\gamma (x-x_c)) -b(x,y) \right),\\
&u=v=0,
\end{aligned}
\end{equation}
where the parameters are set to $A=0.064m$, $x_c = 2.5m$, $h_0 = 0.32m$, $\gamma = \sqrt{\frac{3A}{4h_0}}$. The bottom topography is a cone and defined by
\begin{equation}
b(x,y) = 0.93 \left(1-\frac{r}{r_c}\right)	\qquad \text{ if } r\leq r_c
\end{equation}
with $r=\sqrt{(x-x_c)^2 + (y-y_c)}$,  $r_c = 3.6m$ and $(x_c,y_c) = (12.5,15)$. The domain is $\Omega = [0,25] \times [0,30]$ and bounded by solid wall boundary conditions.
\subsection{TESTCASE 35: 1D Bowl (2D Version)}
\subsection{TESTCASE 43: Curved Dam Break PARTIAL}
Note: interior boundary conditions for partial dam are hacked in Mesh.cpp
\subsection{TESTCASE 44: Curved Dam Break FULL}
\subsection{TESTCASE 45: Curved Dam Break Setup - NO BREAK (WB test)}
Note: interior boundary conditions for full dam are hacked in Mesh.cpp



\end{document} 